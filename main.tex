\documentclass{beamer}
\usepackage{listings}
\usepackage{color}
\usepackage{multirow}
\usepackage{tikz}
\usetikzlibrary{positioning}
\usetikzlibrary{decorations.text}
\usetikzlibrary{decorations.pathmorphing}

\setbeamersize{text margin left=10pt,text margin right=10pt}
%\usetheme{Dresden}
\lstdefinestyle{base}{
  language=c++,
%  emptylines=1,
%  breaklines=true,
%  basicstyle=\ttfamily\color{black},
  moredelim=**[is][\color{red}]{@}{@},
  moredelim=**[is][\color{blue}]{~}{~},
}
\title{GridTools Example Implementation}
\author{}
\date{ }

\begin{document}

\maketitle

\section{Storage Info Example}
\begin{frame}[fragile]
  \frametitle{Storage Info Example}
  we access a storage using meta-information, i.e. strides, data layout, indices

  Optimizations:
  \begin{itemize}
  \item This information might be shared among several storages;
  \item Sometimes can be compile-time information;
  \end{itemize}

  Solution: described in the example \verb1main_storage.cpp1
  \begin{itemize}
  \item create a generic \verb1storage_info1 class containing the meta-information;
  \item constant expression when needed (strides and index computations happen at compile-time)
  \item shared among several storage instances
  \item we will generalize the storage layout with a \verb1layout_map1 object
  \end{itemize}
\end{frame}

\section{Tuple Example}
\begin{frame}[fragile]
\frametitle{Tuple Example}
  we need a tuple of objects and we want to initialize only some of them, in arbitrary order,
  with a specific interface
  {\footnotesize
  \begin{lstlisting}[style=base]
    my_tuple<int, double, std::string>(arg<3>("ciao")
                                       , arg<1>(4))
  \end{lstlisting}
  }
  Problems:
  \begin{itemize}
    \item we want the tuple to be a constant expression when necessary
    \item need to generalize the implementation to generic data types
    \item need to implement a simpler interface when the tuple is actually a set of objecs of the same type,
      i.e.
      {\footnotesize
      \begin{lstlisting}[style=base]
          make_tuple<int, 5>(arg<3>(8))
      \end{lstlisting}
      }
        is
        {\footnotesize
        \begin{lstlisting}[style=base]
          my_tuple<int, int, int, int, int>(arg<3>(8))
        \end{lstlisting}
        }

  \end{itemize}

  Solution: in example \verb1main_tuple.cpp1
\end{frame}
\end{document}
